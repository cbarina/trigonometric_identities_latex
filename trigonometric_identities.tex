\documentclass[a4paper,twoside, 11pt]{article}

% Packages
% ---------------------

\usepackage{amsmath} % Needed for command eqref
\usepackage{amssymb} % For math symbols like R \mathbb{R}
\usepackage{amsthm} % Used to show the proof box
\usepackage{bm} % Bold math text \bm{x}
\usepackage[makeroom]{cancel} % Cross out terms in equations
\usepackage{siunitx} % Physical numbers and units
\usepackage{fancyhdr} % Head and foot options
\usepackage{graphicx} % Embed figures
\usepackage{hyperref} % Uses automatic references \autoref
\usepackage[tableposition=top]{caption} % Place captions on top of the tables
\usepackage{titletoc} % Style table of contents with dots
\usepackage{pgfplots} % Drawing graphsrestrict y to domain=-20:20
\usepackage{enumitem} % Items with custom descriptions instead of numbers/bullets. Example: \begin{description} ... \item [label] text
\usepackage{mathtools} % Used to show the absolute value symbol: \abs{a}

% Used for long division
\usepackage{scalerel}
\usepackage{stackengine}
\usepackage{xcolor}


% Remove the warning for bibliography text:
%
%   "nderfull \hbox (badness 4492) in paragraph at lines"
%
% http://tex.stackexchange.com/a/10928/101976
\usepackage{etoolbox}
\apptocmd{\sloppy}{\hbadness 10000\relax}{}{}

% Remove the warning for bibliography text:
%
%   "Overfull \hbox (5.53845pt too wide) in paragraph at lines"
%
% http://tex.stackexchange.com/questions/171999/overfull-hbox-in-biblatex
\emergencystretch=1em


% Page settings
% ---------------------

\usepackage[top=2cm, bottom=2.5cm,left=2.5cm,right=2.5cm]{geometry} %  Page margins
\setlength{\parskip}{\baselineskip} % Add space between paragraphs
\setlength{\intextsep}{20pt plus 2.0pt minus 2.0pt} % Add vertical space before and after tables and figures (http://tex.stackexchange.com/a/26522/101976)
\parindent=0cm % Remove the paragraph indent of the first line
\addtolength{\jot}{2\jot} % Double the line between equations
\def\arraystretch{1.5} % Increase the vertical table cell margin
\dottedcontents{part}[0em]{\bfseries}{0em}{1pc} % Use dots in the table of contents
\renewcommand{\contentsname}{\centering Contents} % Center the table of contents header
\raggedbottom % Prevents spreading the page content vetically for non-full pages.
\setlength{\headheight}{20pt} % Height of the header
\pgfplotsset{compat=1.12} %

% Used for long division
\newcommand\showdiv[1]{\overline{\smash{\hstretch{.5}{)}\mkern-3.2mu\hstretch{.5}{)}}#1}}
\newcommand\ph[1]{\textcolor{white}{#1}}

% PDFPlots
%
% Empty and filled dots for pdfplot
%
%   Usage:
%     \addplot[holdot] coordinates{(5,0.0909)};
%
\pgfplotsset{soldot/.style={color=black,only marks,mark=*}}
\pgfplotsset{holdot/.style={color=black,fill=white,only marks,mark=*,mark options={scale=1.5, fill=white}}}


% PDFPlots
%
% Vertical asymptote
%
%   Usage:
%     \begin{axis}[vasymptote=1]
%
\pgfplotsset{vasymptote/.style={
    before end axis/.append code={
        \draw[densely dashed] ({rel axis cs:0,0} -| {axis cs:#1,0})
        -- ({rel axis cs:0,1} -| {axis cs:#1,0});
    }
}}

% PDFPlots
%
% Horizontal asymptote
%
%   Usage:
%     \begin{axis}[hasymptote=1]
%
\pgfplotsset{hasymptote/.style={
    before end axis/.append code={
        \draw[densely dashed] ({axis cs:0,#1} -| {rel axis cs:0,0})
        -- ({axis cs:0,#1} -| {rel axis cs:1,0});
    }
}}

% Set vertical space around equations.
\AtBeginDocument{%
 \abovedisplayskip=15pt plus 5pt minus 5pt
 \abovedisplayshortskip=12pt plus 3pt
 \belowdisplayskip=15pt plus 5pt minus 5pt
 \belowdisplayshortskip=12pt plus 3pt minus 4pt
}


% Custom commands
% ----------------------

% Assignment answer header
% Example:
% 	\answer{Q.1 (b)}
% 	Result: Answer Q.1 (b)
% ----------------------

\newcommand{\answer}[1]{
	{
		\medskip
		\large\textbf{Answer #1}
	}
}

% Make the proof white box a black box.
\renewcommand{\qedsymbol}{\rule{0.7em}{0.7em}}

% Show black box on the right
\newcommand{\myproof}{
  {
    \begin{flushright}
      \qedsymbol
    \end{flushright}
  }
}

% Absolute value symbol. Usage: \abs{a}
\DeclarePairedDelimiter\abs{\lvert}{\rvert}%

% Title page

\title{Triginometric Identities}
\author{Evgenii Neumerzhitckii}
\date{May 17, 2016}

\begin{document}

\maketitle
\thispagestyle{empty} % Remove page number from title page

\pagebreak



% Parts
% ---------------------


\section*{Basic trigonometric identities}

\section*{Common angles}


\begin{table}[!ht]
\setlength{\tabcolsep}{1em} % Increase the horizontal table cell margin
\centering
  \tabulinesep=1.5mm
  \begin{tabu}{|c|c|c|c|c|c|c|c|}
    \hline
    \textbf{Degrees}
      & $\SI{0}{\degree}$
      & $\SI{30}{\degree}$
      & $\SI{45}{\degree}$
      & $\SI{60}{\degree}$
      & $\SI{90}{\degree}$\\
    \hline
    \textbf{Radians}
      & $\displaystyle 0$
      & $\displaystyle \frac{\pi}{6}$
      & $\displaystyle \frac{\pi}{4}$
      & $\displaystyle \frac{\pi}{3}$
      & $\displaystyle \frac{\pi}{2}$\\
    \hline
    \textbf{\bm{$\sin \theta$}}
      & $\displaystyle 0$
      & $\displaystyle \frac{1}{2}$
      & $\displaystyle \frac{\sqrt 2}{2}$
      & $\displaystyle \frac{\sqrt 3}{2}$
      & $\displaystyle 1$\\
    \hline
    \textbf{\bm{$\cos \theta$}}
      & $\displaystyle 0$
      & $\displaystyle \frac{\sqrt 3}{2}$
      & $\displaystyle \frac{\sqrt 2}{2}$
      & $\displaystyle \frac{1}{2}$
      & $\displaystyle 0$\\
    \hline
    \textbf{\bm{$\tan \theta$}}
      & $\displaystyle 0$
      & $\displaystyle \frac{\sqrt 3}{3}$
      & $\displaystyle 1$
      & $\displaystyle \sqrt 3$
      &\\
    \hline
  \end{tabu}
\end{table}
\subsection*{Reciprocal functions}

\begin{align*}
  \cot x &= \frac{1}{\tan x}\\
  \csc x &= \frac{1}{\sin x}\\
  \sec x &= \frac{1}{\cos x}
\end{align*}
\subsection*{Pythagorean identities}

\begin{align*}
  \sin^2 \theta + \cos^2 \theta &= 1\\
  1 + \tan^2 \theta &= \sec^2 \theta\\
  1 + \cot^2 \theta &= \csc^2 \theta
\end{align*}
\subsection*{Cofunction identities}

\begin{align*}
  \sin(\frac{\pi}{2} - \theta) &= \cos \theta\\
  \cos(\frac{\pi}{2} - \theta) &= \sin \theta\\
  \tan(\frac{\pi}{2} - \theta) &= \cot \theta\\
  \cot(\frac{\pi}{2} - \theta) &= \tan \theta\\
  \sec(\frac{\pi}{2} - \theta) &= \csc \theta\\
  \csc(\frac{\pi}{2} - \theta) &= \sec \theta
\end{align*}
\subsection*{Sum and difference of angles}

\begin{align*}
  \sin(x + y) &= \sin x \cos y + \cos x \sin y\\
  \sin(x - y) &= \sin x \cos y - \cos x \sin y\\
  \cos(x + y) &= \cos x \cos y - \sin x \sin y\\
  \cos(x - y) &= \cos x \cos y + \sin x \sin y\\
  \tan(x + y) &= \frac{\tan x + \tan y}{1 - \tan x \tan y}\\
  \tan(x - y) &= \frac{\tan x - \tan y}{1 - \tan x \tan y}
\end{align*}
\subsection*{Double angles}

\begin{align*}
  \sin(2\theta)  &= 2 \sin \theta \cos \theta\\
  \cos(2\theta)  &= \cos^2 \theta - \sin^2 \theta\\
            &= 2 \cos^2 \theta - 1\\
            &= 1 - 2 \sin^2 \theta\\
  \tan(2\theta)  &= \frac{2 \tan \theta}{1 - \tan^2 \theta}
\end{align*}



\end{document}